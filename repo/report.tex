\documentclass[12pt,a4j,dvipdfmx]{jarticle}
\usepackage[dvipdfmx]{graphicx} % 画像を読み込むためのパッケージ
\usepackage{here} % 図を指定した位置に強制的に配置するためのパッケージ
\usepackage{amsmath} %数学記号を拡充するパッケージ
\renewcommand\thefootnote{注 \arabic{footnote}} % 脚注の番号表記をロマン数字に設定するためのおまじない
\usepackage{hhline} % 二重線に関するパッケージ
\renewcommand{\thesection}{課題\arabic{section}}
%\renewcommand{\thesubsection}{1-\Alph{subsection}}
%\renewcommand{\thesubsection}{(\arabic{subsection})}
\renewcommand{\thesubsubsection}{\arabic{section}.\arabic{subsubsection}}
\renewcommand{\thesubsection}{}

\renewcommand{\labelenumi}{(\arabic{enumi})}
\usepackage{tikz}
\usetikzlibrary{matrix,calc}
%\usepackage{listings,jlisting}

%\lstset{
%    basicstyle={\ttfamily\footnotesize}, %書体の指定
%    frame=single, %フレームの指定
   % framesep=10pt, %フレームと中身(コード)の間隔
    %breaklines=true, %行が長くなった場合の改行
    %linewidth=12cm, %フレームの横幅
   % lineskip=-0.5ex, %行間の調整
%}

\begin{document}
\title{プログラミング言語実験  Schemeレポート\\} 
\author{1710199\\ 君島 雄一郎}
\date{}

\maketitle
\thispagestyle{empty}

\newpage
%---------------------------------------------------
\section{}%kadai1
\subsection{設計上の留意点}

\subsection{出力例}
\subsubsection{関数map-tree}%1-1
\subsubsection{関数map-tree2}%1-2


\subsection{考察}


\newpage
%---------------------------------------------------
\section{}%kadai2
\subsection{設計上の留意点}


\subsection{出力例}
\subsubsection{関数get-depth}%2-1
\subsubsection{関数get-cousin}%2-2


\subsection{考察}


\newpage
%---------------------------------------------------
\section{}%kadai3
\subsection{設計上の留意点}

\subsection{出力例}
\subsubsection{関数diff}%3-1
\subsubsection{関数tangent}%3-2
\subsubsection{関数diff2}%3-3
\subsubsection{関数simple}%3-4


\subsection{考察}


\end{document}
